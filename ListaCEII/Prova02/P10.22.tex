
\section*{Problema P10.22}

\renewcommand*\thesection{10.22}
\numberwithin{equation}{section}

\begin{center}
    \includegraphics[scale=1.0]{P10.22.jpg}
\end{center}

\subsection*{(a)}

Temos as seguintes informações dadas:

\[ S_1 = 10000 + j4000 \un{VA} \quad , \quad Z_2 = 60 + j80 \;\Omega \quad , \quad V_1 = V_2 = 1000\fase{0} \un{V} \]

Em $L_1$, temos    

\[ S_1 = V_1 \cdot (I_1)^* \quad \Rightarrow \quad I_1 = \left(\frac{S_1}{V_1}\right)^* \quad \Rightarrow \quad I_1 = 10 -j4 \un{A}\]

Uma vez calculado $I_1$, agora calculamos a impedância da carga $L_1$.

\[ Z_1 = \frac{V_1}{I_1} = \frac{1000}{10 - j4} = 86.2 + j34.5 \;\Omega \]

Agora vamos calcular a corrente $I_2$ da carga $L_2$.

\[ I_2 = \frac{V_2}{Z_2} = \frac{1000}{60 + j80} = 6 - j8 \un{A}  \]

Assim, a corrente fornecida pela fonte é

\[ I_g = I_1 + I_2 = 10 -j4 +  6 - j8 \un{A} = 16 - j12 \un{A} \]

A impedância $Z_{in}$ vista pela fonte é

\[ Z_{in} = 0.5 + j0.05 + (Z_1 \; // \; Z_2) \]

\[ Z_{in} = 0.5 + j0.05 + \frac{1}{\frac{1}{86.2 + j34.5} + \frac{1}{60 + j80}} \]

\[ Z_{in} = 0.5 + j0.05 + 40 + j30 = 40.5 + j30.05 \;\Omega \]

Finalmente, a tensão da fonte é

\[ V_g = Z_{in} \cdot I_g = (40.5 + j30.05)(16 - j12) = 1008.6 - j5.2 \un{V}\]

\[ \boxed{V_g = 1008.6 \fase{-0.295}  \un{V}} \]

\subsection*{(b)}

Usando proporcionalidade (regra de três simples), temos   

\[ \frac{T}{\Delta t} = \frac{360^{\circ}}{\Delta \phi}  \]

onde $\Delta \phi$ é a diferença de fase entre os sinais. Portanto, usando $T = \frac{2\pi}{\omega}$,

\[ \Delta t = \frac{2\pi}{\omega} \frac{\Delta \phi}{360^{\circ}} \]

Substituindo,

\[ \Delta t = \frac{2\pi}{100\pi} \frac{0.295^{\circ}}{360^{\circ}} \]

\[ \boxed{\Delta t = 16.39 \;\mu\textrm{s}} \]

\subsection*{(c)}

$V_L$ está $\Delta \phi = 0.295^{\circ}$ adiantada em relação a $V_g$.









