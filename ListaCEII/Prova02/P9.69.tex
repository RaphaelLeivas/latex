\section*{Problema P9.69}

\renewcommand*\thesection{9.69}
\numberwithin{equation}{section}

\begin{center}
    \includegraphics[scale=1.0]{P9.69.jpg}
\end{center}

O primeiro passo é expressar $V_o$ em função das tensões de entrada no AmpOp.
Aplicamos análise nodal no nó $(B)$.

\[ i_- + i_+ + i_{GND} + \frac{V_B - V_o}{R_s} + \frac{V_B - V_A}{R_2} = 0 \]

Como o Amplificador Operacional é ideal, temos

\begin{equation}\label{eq:9.69.1}
    i_- = i_+ = 0 \; \textrm{A}
\end{equation}

Além disso, temos $V_B = 0$. Substituindo na expressão do nó, temos

\[ i_{GND} + \frac{V_o}{R_s} + \frac{V_A}{R_2} = 0 \]

Isolando $V_o$, temos

\begin{equation}\label{eq:9.69.2}
    V_o = - R_s\left(i_{GND} + \frac{V_A}{R_2}\right)
\end{equation}

Agora aplcamos análise de malhas, com as correntes de malha $i_1$ e $i_2$ na figura. Note que $i_2 = i_{GND}$. \\
Começamos pela malha 1:

\[ -V_g + R_1i_1 + \frac{1}{j\omega C}(i_1 - i_2) = 0 \]

\[ R_1i_1 - \frac{j}{\omega C}(i_1 - i_2) = V_g \]

\[ i_1\left(R_1 - \frac{j}{\omega C}\right) + i_2\left(- \frac{j}{\omega C}\right) = V_g \]

Agora vamos para a malha 2:

\[ \frac{1}{j\omega C}(i_2 - i_1) + R_2i_2 = 0 \]

\[ i_1\left(- \frac{j}{\omega C}\right) + i_2\left(R_2 + \frac{j}{\omega C}\right) = 0 \]

Com as duas equações de malha, temos o sistema linear

\begin{equation}\label{eq:9.69.3}
    \begin{bmatrix}
        R_1 - \frac{j}{\omega C} & - \frac{j}{\omega C} \\
        - \frac{j}{\omega C} & R_2 + \frac{j}{\omega C}
    \end{bmatrix}
    \begin{bmatrix}
        i_1 \\
        i_2
    \end{bmatrix}
    =
    \begin{bmatrix}
        V_g \\
        0
    \end{bmatrix}
\end{equation}


















