\section*{Questão 1 (d)}

Começamos no caminho certo invertendo a ordem de integração e separando a integral em duas na soma:

\[ \int_{2}^{4} \int_{-1010}^{1010} \left[y^5e^{x^2+y^2} + 1\right] \, dy \, dx \]

\[ \int_{2}^{4} \int_{-1010}^{1010} y^5e^{x^2+y^2} \, dy \, dx + \int_{2}^{4} \int_{-1010}^{1010} \, dy \, dx \]

Observe que a integral da esquerda é do tipo separável: os extremos de integração são constantes 
e podemos separar o integrando em apenas funções de x de um lado, e apenas funções de y do outro 

\[ \int_{2}^{4} \int_{-1010}^{1010} \left[y^5 e^{y^2}\right] \left[e^{x^2}\right] \, dy \, dx + \int_{2}^{4} \int_{-1010}^{1010} \, dy \, dx \]

\[ \left[\int_{-1010}^{1010} y^5 e^{y^2} \, dy\right] \left[\int_{2}^{4} e^{x^2} \, dx\right]  + \int_{2}^{4} \int_{-1010}^{1010} \, dy \, dx \]

Note que $e^{y^2}$ é uma função par, e $y^5$ é uma função ímpar. Multiplicando função par por função ímpar, 
o resultado é uma função ímpar. \\

Logo,  $y^5 e^{y^2}$ é uma função ímpar.  \\

Quando integramos uma função ímpar em um intervalo simétrico em relação à origem (no caso, de $-1010$ a $1010$), o resultado é zero, pois a área 
do lado esquerdo da origem "cancela" com a área do lado da direita. Assim, a expressão fica  

\[ 0 \left[\int_{2}^{4} e^{x^2} \, dx\right]  + \int_{2}^{4} \int_{-1010}^{1010} \, dy \, dx \]

Uma outra maneira de ver que a integral da esquerda da zero, sem usar paridade, é usar substituição.

\[ \int_{-1010}^{1010} y^4 y e^{y^2} \, dy \]

Seja $u = y^2$, logo $du = 2y \, dy$

\[ \frac{1}{2} \int_{-1010}^{1010} u^2 e^{u^2} \, du \]

Agora usamos integração por partes:

\[ \frac{1}{2} \left[ e^{u^2} \frac{u^3}{3} - \int 2ue^{u^2} \frac{u^3}{3} \right] \, du \]

Para aqui. Fica muito complicado.



