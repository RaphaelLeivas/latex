\documentclass[oneside,12pt]{article}
\usepackage[a4paper,width=150mm,top=25mm,bottom=25mm]{geometry}
\usepackage[utf8]{inputenc}
\usepackage[english]{babel}
\usepackage{graphicx}   \graphicspath{{figures/}}
\usepackage[margin=30pt,labelfont=bf,labelsep=period,font={small,stretch=1.1}]{caption} % Caption settings
\usepackage{setspace}
\usepackage[font={normal,stretch=1.1}]{subcaption}
\usepackage{amssymb,wasysym}% Symbols
% \usepackage[table,dvipsnames]{xcolor}     % Provides a common set of commands for color manipulation
\usepackage[hidelinks]{hyperref} % Links: table of contents, references, bibliography, figures, pages, url's, etc.
\usepackage{fancyhdr} \pagestyle{fancy} % Page heading
\fancyhead[L]{\nouppercase{\rightmark}} \fancyhead[R]{\nouppercase{\leftmark}} % No uppercase
\usepackage{array} % To control table cell width
\usepackage{makecell}
\usepackage[]{natbib} \bibliographystyle{apalike}    % Bibliography management
\usepackage{blindtext}
\usepackage{soul}
\usepackage{lscape}
\usepackage{color}
\usepackage{algpseudocode}
\usepackage{algorithm}
\newcommand{\un}[1]{\,\mathrm{#1}}
\algnewcommand{\LineComment}[1]{\State \(\triangleright\) #1}
\newcommand{\bigO}{\mathcal{O}}
\newcommand{\codeword}[1]{\texttt{\textcolor{blue}{#1}}}
\begin{document}

\setlength{\parindent}{0 em}    % Paragraph indentation
\setlength{\parskip}{0.7 em}    % Space between paragraph and preceding text
\renewcommand{\baselinestretch}{1.5}    % Line spacing

\newcommand{\ProjectTitle}{AMESP - Phase 2 WP3} % <---Write here the project title!!!

\begin{titlepage}
    \begin{center}
    \vspace*{3 cm}
    \Huge
    \textbf{Report} \\
    \vspace{1 cm}    
    \textbf{\ProjectTitle}
    \\
    \vspace{1 cm}
    \vfill      % Automatically add in the amount of vertical space needed for the content to fill the page
    \includegraphics[width=60mm]{figures/NHLStenden-logo.png} \\
    \end{center}
    \vspace{2 cm}
    \normalsize
    Title : \ProjectTitle
    \\
    Authors : K. S. Moreira, Luewton L F Agostinho, R.H. Braga Leivas
    \\
    Date : \today 
    \\
    Code : 
    \\
    Version : 1 \\
    Status : draft \\
    Mailing list : 
    \\
    Copy to : \\
    Classification : confidential
\end{titlepage}

% \chapter*{} \thispagestyle{empty}   % "Page break"
% \chapter*{} \thispagestyle{empty}   % "Page break"

% ---------- MAIN TEXT ---------- %
\setlength{\parindent}{0 em}  % Paragraph indentation
\setlength{\parskip}{0.7 em}    % Space between paragraph and preceding text

\newpage    \pagestyle{empty}
\tableofcontents
\pagestyle{fancy}
\newpage

% ==============================================
% \newpage
\section{WP3 Overview}

\subsection{Objectives}

As stated on the proposal, we have two main goals in the WP3:

\begin{enumerate}
    \item Investigate methods to identify the electrospray mode during operation by reading current values
    \item Investigate methods to perform corrective actions to restore the electrospray operation by sending 
    commands to the pump and/or power supply
\end{enumerate}

We have explored two approaches that could achieve these goals, and they are detailed on Sections \ref{sec:current-based} and
\ref{sec:pid}. For each approach, we discuss how they can be implemented in the system developed by Gilbert,
their advantadges and disadvantages, and how well they perform in achieving the above goals. 

Section \ref{sec:utils} shows some results obtained during the control investigations that were not directly used to achieve
the goals of this Work Package, but we believe they could still prove useful to Gilbert and provide inputs to how the
multinozzle prototype operates.

\subsection{Timeline of Executed Tasks}

\autoref{fig:timeline} shows a timeline of executed tasks for Work Package 3.

\begin{figure}[h!]
    \centering
    \includegraphics[width=\textwidth,trim=1 1 1 1,clip]{figures/timeline.png}
    \caption{Timeline of executed tasks.}
    \label{fig:timeline}
\end{figure}

The tasks on \autoref{fig:timeline} relate directly to the sections of this report in the table of contents. 
The gap between August and September 
corresponds to the holiday period of several members of the project. Tasks executed of other Work Packages are not included in 
the timeline.

\subsection{Conventions}

Throughout this report, we'll refer multiple times to physical variables of the system. \autoref{fig:setup-conventions}
shows the variable conventions we'll use in this text.

\begin{figure}[h!]
    \centering
    \includegraphics[width=.9\textwidth,trim=1 1 1 1,clip]{figures/setup-conventions.png}
    \caption{Variable conventions and nomenclature. Source: adapted from \cite{Verdoold2013}.}
    \label{fig:setup-conventions}
\end{figure}

As shown in \autoref{fig:setup-conventions}, we have

\begin{itemize}
    \item $i_N$: current flowing from the positive high voltage source to the nozzles
    \item $i_C$: current flowing from the crown to the negative high voltage source
    \item $i_{GND}$: current flowing from the ground to the ring
    \item $\phi$: flow rate of the syringe pump
    \item $V_+$: voltage of the positive high voltage source
    \item $V_-$: voltage of the negative high voltage source
\end{itemize}

The direction of the currents was chosen as shown in \autoref{fig:setup-conventions} to ensure they are always positve 
in the measurements, facilitating the analysis.

% ==============================================
% \newpage
\section{Current-based Classification and Control}\label{sec:current-based}

The general strategy adopted in this approach is to first classify the electrospray mode by measuring the current values 
on the system, and then experimentally design a controller that can move from an intermittent
to a cone-jet spray mode. 

The classification method is based on the analysis made by \cite{Verdoold2013}. However, Verdoold's method was designed for the 
single nozzle, and it is not clear if we can 
extend his classification to a multinozzle configuration. Therefore, part of this work 
includes an attempt to extend his classification method to the system developed by 
Gilbert.

\subsection{Measuring the currents by spray mode}

The first test done was to measure the current on all three lines of the sprayhead and 
verify if we see a pattern in the shape of current that can be used to classify the spray mode.
\autoref{fig:setup-three-currents-spray-mode} shows the setup used for this test. $V_-$ was fixed on 
$V_- = - 4.5 \un{kV}$.

\begin{figure}[h!]
    \centering
    \includegraphics[width=\textwidth,trim=1 1 1 1,clip]{figures/setup-three-currents-spray-mode.png}
    \caption{Setup to measure all three currents on the sprayhead. Source: adapted from \cite{Verdoold2013}.}
    \label{fig:setup-three-currents-spray-mode}
\end{figure}

Using three oscilloscopes, all three currents were sampled at 5 kHz, collecting 20.000 samples of each
(totalling a 4 seconds time window). Notice that, although the oscilloscope has multiple channels, 
we cannot use the same oscilloscope as the channels are interconnected internally: the high voltage 
differences would damage the instrument. Therefore, we use one oscilloscope for each line. 

Since the oscilloscopes are in series, they will measure the voltage drop $\Delta V$ accross the instrument. We know from the manufacturer's 
datasheet that the impedance of the oscilloscope is $Z_{in} = 2 \un{M} \Omega$. Threfore, using Ohm's Law, we can measure the 
current flowing through the oscilloscope with $i = \frac{\Delta V}{Z_{in}}$. 

Furthermore, the addition of the oscilloscope in series does not cause a significant impact on the operation of the sprayhead.
For instance, when we have a cone-jet at $i_N = 3 \un{uA}$ and $V_+ = 5 \un{kV}$, the impedance of the sprayhead $Z_{sh}$,
as seen from the source $V_+$, is

\[  Z_{sh} = \frac{V_+}{i_N} = \frac{5 \un{kV}}{3 \un{uA}} = 1.6 \un{G} \Omega \]

Therefore, $Z_{sh}$ at least 3 orders of magnitude larger than $Z_{in}$, making the addition of the oscilloscope in series
have a negligible impact on the operation of the sprayhead.

We use the HS camera to identify the EHDA spray mode of the nozzles.
 
\autoref{fig:three-currents-spray-mode-50ms} and \autoref{fig:three-currents-spray-mode-500ms} show the waveforms 
obtained for $\phi = 20 \un{mL/h}; \phi = 30 \un{mL/h}; \phi = 50 \un{mL/h}$ for two different time windows: 50ms 
and 500ms.

\begin{landscape}
    \begin{figure}[h!]
        \centering
        \includegraphics[width=0.75\paperheight,trim=1 1 1 1,clip]{figures/three-currents-spray-mode-50ms.png}
        \caption{$i_N$, $i_{GND}$ and $i_C$ over time by different spray modes - 50 ms time window.}
        \label{fig:three-currents-spray-mode-50ms}
    \end{figure}
\end{landscape}

\begin{landscape}
    \begin{figure}[h!]
        \centering
        \includegraphics[width=0.75\paperheight,trim=1 1 1 1,clip]{figures/three-currents-spray-mode-500ms.png}
        \caption{$i_N$, $i_{GND}$ and $i_C$ over time by different spray modes - 500 ms time window.}
        \label{fig:three-currents-spray-mode-500ms}
    \end{figure}
\end{landscape}

As we can see in \autoref{fig:three-currents-spray-mode-50ms} and \autoref{fig:three-currents-spray-mode-500ms}, 
we don't see a clear distinction in the shape of the current 
by different spray modes, in both time windows. The sharp peaks we see on the waveforms are most likely caused
by a high-frequency noise in signal, especially since we use banana adapters for the oscilloscope in this test, instead
of coaxial cables which pick up less noise.

Attempting to see a distinction on \autoref{fig:three-currents-spray-mode-500ms} via the standard deviation also
fails, as shown on \autoref{fig:three-currents-spray-mode-sd}.

\begin{figure}[h!]
    \centering
    \includegraphics[width=1\textwidth,trim=1 1 1 1,clip]{figures/three-currents-spray-mode-sd.png}
    \caption{Standard deviation on \autoref{fig:three-currents-spray-mode-500ms} of $i_N$, $i_{GND}$ and $i_C$
    by spray mode.}
    \label{fig:three-currents-spray-mode-sd}
\end{figure}

Without a clear distinction in the current waveforms it would not possible to continue with this approach.
Therefore, we need to first understand why we are not seing distinctions in the shape 
of the current, particularly between the intermittent and cone-jet spray modes, as it is clear in the 
literature that there should be a difference.

To do this, we'll begin by attempting to reproduce \cite{Verdoold2013}'s results, with the goal of isolating 
if the problem is our measurement strategy or if it is something related to the sprayhead itself.

\subsection{Reproducing Results from \cite{Verdoold2013}}

\autoref{fig:verdoold-reproduce-setup} shows the setup used to reproduce \cite{Verdoold2013}'s approach.
We used a sampling frequency of 5 kHz and $\phi = 1 \un{mL/h}$. 

\begin{figure}[h!]
    \centering
    \includegraphics[width=.8\textwidth,trim=1 1 1 1,clip]{figures/verdoold-reproduce-setup.png}
    \caption{Setup used to reproduce Verdoold's classification method. Source: adapted from \cite{Verdoold2013}.}
    \label{fig:verdoold-reproduce-setup}
\end{figure}

The results obtained are shown in \autoref{fig:verdoold-reproduce-results}. 

\begin{figure}[h!]
    \centering
    \includegraphics[width=1\textwidth,trim=1 1 1 1,clip]{figures/verdoold-reproduce-results.png}
    \caption{Results from the attempt to reproduce Verdoold's results on our setup.}
    \label{fig:verdoold-reproduce-results}
\end{figure}

As we can see in \autoref{fig:verdoold-reproduce-results}, we see a clear distinction between the spray modes, 
which is what we wish to see in the multinozzle. We can conclude that our measurement methodology can 
reproduce the results, therefore it must be something in the multinozzle that is "hiding" the intermittent spray 
signal.

Comparing the single nozzle setup on \autoref{fig:verdoold-reproduce-setup} and the multinozzle on \autoref{fig:setup-three-currents-spray-mode}, 
the most significant difference is indeed the presence of the crown. Therefore, let's begin by understanding the influence 
of the crown on $i_{GND}$, which is the current that we know is capable of showing a distinction of spraying 
modes. 

\subsection{Crown influences on $i_{GND}$}

To understand the influence of the crown on the ground current, we'll use the same setup shown on 
\autoref{fig:setup-three-currents-spray-mode}, but we'll make $V_+ = 0 \un{V}$, $\phi = 0 \un{mL / h}$ and measure $i_{GND}$ for different
crown voltages. Since there is no flow 
and no positive voltage, we'll be measuring the current on the ground ring introduced by the crown only.
In addition, we'll use coaxial cables during the measurements.

\autoref{fig:ignd-by-crown-no-filters} shows the shape of $i_{GND}$ for different values of $V_-$.

\begin{figure}[h!]
    \centering
    \includegraphics[width=1\textwidth,trim=1 1 1 1,clip]{figures/ignd-by-crown-no-filters.png}
    \caption{$i_{GND}$ for different values of $V_-$. (a) Waveform and (b) standard deviation}
    \label{fig:ignd-by-crown-no-filters}
\end{figure}

As seen of \autoref{fig:ignd-by-crown-no-filters} (a), the crown alone introduces a signal on $i_{GND}$ starting
from $V_- = - 4 \un{kV}$, which increases in both average value and standard deviation as $V_-$ increases.
This is consistent with was happens at the crown: from $V_- = - 4 \un{kV}$ onwards the sharp needles 
of the crown begin to ionize the air, producing ions that can be 
directed to ground ring. This results in a current $i_{GND} > 0$ induced by the crown.

In addition, on \autoref{fig:ignd-by-crown-no-filters} (b) we see that the standard deviation introduced by the crown
is significant. As we saw on \autoref{fig:verdoold-reproduce-results}, the intermittent spray mode 
displays peaks in the current signal in order of 50 nA, but the "noise" introduced by the 
crown alone is already over 50 nA when $V_- = -4.5 \un{kV}$, which is voltage used on the results 
of \autoref{fig:three-currents-spray-mode-500ms}. Therefore, it is reasonable to assume that the reason we are
not seing a good distinction between the spray modes on \autoref{fig:three-currents-spray-mode-500ms} is 
because of this signal introduced by the crown alone.

\subsubsection{Atenuating Crown Influences on $i_{GND}$}

In order to verify the above hypothesis, we can try to reduce the influence of the crown
in the signal and verify if the intermittent signal becomes distinguishable. We can achieve this adding digital filters 
in the oscilloscope software to remove the following frequencies:

\begin{itemize}
    \item 50 Hz frequency from the electric grid: use a stop band in the range 48 - 52 Hz
    \item All frequencies above 100 Hz: use low pass filter with cut-off frequency 100 Hz.
\end{itemize}

Since the intermittent peaks are usually under 100 Hz \citep{Verdoold2013}, we can remove 
everything above this frequency from the signal as it is not what we wish to measure.

Using this, we once again collect the signals of \autoref{fig:ignd-by-crown-no-filters}, obtaining the signal 
on \autoref{fig:ignd-by-crown-with-filters}.

\begin{figure}[h!]
    \centering
    \includegraphics[width=1\textwidth,trim=1 1 1 1,clip]{figures/ignd-by-crown-with-filters.png}
    \caption{$i_{GND}$ for different values of $V_-$ with digital filters. (a) Waveform and (b) standard deviation}
    \label{fig:ignd-by-crown-with-filters}
\end{figure}

Comparing \autoref{fig:ignd-by-crown-with-filters} and \autoref{fig:ignd-by-crown-no-filters}, we see that the filters significantly 
reduce the standard deviation (i.e. the "noise") in $i_{GND}$. For $V_- = - 4.5 \un{kV}$, the standard deviation is already almost 
three times smaller. However, it remais above 50 nA, so it could still make it difficult to see the intermittent peaks in the signal.
Ideally, we would use the $V_- = - 4 \un{kV}$ to get the smallest amount of noise introduced, but it is not clear if it's possible
to achieve a good neutralization with such a small $V_-$.

\subsubsection{8 vs 16 Needles in the Crown}

Around this time, Gilbert request us to test the crown with 8 needles for the WP2, as opposed to the 16 needles we had always used until 
this point. Difficulties to reinsert removed needles meant that from this point onwards we would always use a needle with 8 needles. 

Therefore, before we continue with this analysis, we need to understand how the signal introduced on $i_{GND}$ has changed. We've re-done 
the test of \autoref{fig:ignd-by-crown-with-filters}, obtaining the results shown on \autoref{fig:ignd-by-crown-8-needles}.

\begin{figure}[h!]
    \centering
    \includegraphics[width=1\textwidth,trim=1 1 1 1,clip]{figures/ignd-by-crown-8-needles.png}
    \caption{$i_{GND}$ for different values of $V_-$ with digital filters - Crown with 8 needles. (a) Waveform and (b) standard deviation}
    \label{fig:ignd-by-crown-8-needles}
\end{figure}

Comparing \autoref{fig:ignd-by-crown-with-filters} and \autoref{fig:ignd-by-crown-8-needles}, we see that the 8 needles introduce 
a much smaller signal on the $i_{GND}$, both in terms of average value as in standard deviation. This will be helpful to make  
the intermittent peaks on the signal more visible, as for $V_- = - 4.5 \un{kV}$ the introduced standard deviation is only 10 nA.

\subsection{Classifying the Spraying Mode}

With everything that we've now learned about the influence of the crown and the necessity of filters, we can now repeat the experiment 
of \autoref{fig:setup-three-currents-spray-mode}, using the same setup of \autoref{fig:setup-three-currents-spray-mode}, but again only
measuring $i_{GND}$. The oscilloscope was configured with $f_s = 5 \un{kHz}$ and a sample size of $N_s = 20.000$. The crown had 8 needles 
and was fixed with $V_- = - 4 \un{kV}$ to reduce as much as possible the influence of the crown on $i_{GND}$.

\autoref{fig:ignd-good-signal} shows the result obtained. The intermittent is clearly distinguishable from the cone-jet mode. However, 
as expected, we cannot distinguish the elongated cone-jet and the multi-jet from the cone-jet.

\begin{figure}[h!]
    \centering
    \includegraphics[width=1\textwidth,trim=1 1 1 1,clip]{figures/ignd-good-signal.png}
    \caption{$i_{GND}$ for different spray modes.}
    \label{fig:ignd-good-signal}
\end{figure}

Calculating the standard deviation on \autoref{fig:ignd-good-signal} also allows for a good distinction between the spraying modes,
as shown in \autoref{fig:ignd-good-sd}.

\begin{figure}[h!]
    \centering
    \includegraphics[width=1\textwidth,trim=1 1 1 1,clip]{figures/ignd-good-sd.png}
    \caption{Standard deviation of $i_{GND}$ for different spray modes.}
    \label{fig:ignd-good-sd}
\end{figure}

Based on \autoref{fig:ignd-good-sd}, we could use a simple if-else classification algorithm based on the standard deviation, 
defining a threshold for what is classified as intermittent or cone-jet. For example, we could say that if $\sigma > 60 \un{nA}$
then it is intermittent, else it is cone-jet (once inside the stable region defined by the mapping of the WP1).
An algorithm for this will be further explored on \autoref{sec:classification-algorithm}.


\subsection{Optimizing the Signal Acquisition}

In the previous sections, the signal was acquired using the minimum sampling frequency
suggested by the \cite{Verdoold2013} of $f_s = 5 \un{kHz}$. A sample size of $N_s = 20.000$ was used to obtain a spectral resolution of 0.25 Hz
for frequency domain anaysis, also suggested by Verdoold as the minimum. However, talks with Gilbert showed that the sampling frequency was too
computationally expensive and the sample size was too large, as it resulted in a sampling time window of 4 seconds, which is too slow.

Therefore, to attempt to meet these requirements, we need to find the minimum sampling frequency and
minimum sample size that can still reliably distinguish the signal of the intermittent from the cone-jet.

To do this, we'll use the following method:

\begin{itemize}
    \item Collect a time window of $T = 100$ seconds for different values of $f_s$, resulting in a sample size of $N_s = T \cdot f_s$
    \item Break the $T = 100$ seconds into smaller time windows - denoted as $S_i$ - of size $T_S$. 
    \item Calculate the relevant statistical parameters in each $S_i$ and store these values in an array of size $T / T_S$
    \item Plot a boxplot of the calculated statistical parameters
\end{itemize}

\autoref{fig:subsample-size-strategy} further explains the method above visually. $T_S$ will be chosen for different values so that its influence
can be verified. Note that $i = 1, 2, ..., T / T_S$, and the goal is to find the minimum $T_S$.

\begin{figure}[h!]
    \centering
    \includegraphics[width=.8\textwidth,trim=1 1 1 1,clip]{figures/subsample-size-strategy.png}
    \caption{Method to find the minimum sample size that can still classify the EHDA mode via the current.}
    \label{fig:subsample-size-strategy}
\end{figure}

We'll use the same setup shown in \autoref{fig:setup-three-currents-spray-mode}, using only the oscilloscope for $i_{GND}$.
We begin with $f_s = 5 \un{kHz}$, with $T_S = 0.01 \un{s}; 0.1 \un{s}; 1 \un{s}; 10 \un{s}$. Note that we are first changing
$T_S$ over four orders of magnitude to understand the general influence of $T_S$ on the calculated standard deviation.
The result obtained is shown on \autoref{fig:first-test-subsample}.

\begin{figure}[h!]
    \centering
    \includegraphics[width=\textwidth,trim=1 1 1 1,clip]{figures/first-test-subsample.png}
    \caption{Calculated standard deviation for $f_s = 5 \un{kHz}$, with $T_S = 0.01 \un{s}; 0.1 \un{s}; 1 \un{s}; 10 \un{s}$}
    \label{fig:first-test-subsample}
\end{figure}

As we can see on \autoref{fig:first-test-subsample}, $T_S = 1 \un{s}$ appears to be the best sample size to 
differentiate between the intermittent and the cone-jet. Other orders of magnitude of $T_S$ do not allow for a clear distinction
between spraying mode via the statistical values. The next test is to change $T_S$ around 1 second and compare them, 
using $T_S = 0.25 \un{s}; 0.5 \un{s}; 0.75 \un{s}; 1 \un{s}$. \autoref{fig:second-test-subsample} shows the results obtained  
for this test.

\begin{figure}[h!]
    \centering
    \includegraphics[width=\textwidth,trim=1 1 1 1,clip]{figures/second-test-subsample.png}
    \caption{Calculated standard deviation for $f_s = 5 \un{kHz}$, with $T_S = 0.25 \un{s}; 0.5 \un{s}; 0.75 \un{s}; 1 \un{s}$}
    \label{fig:second-test-subsample}
\end{figure}

As seen on \autoref{fig:second-test-subsample}, $T_S = 0.5 \un{s}$ appears to be the smallest sample size that can differentiate the spraying modes.
$T_S = 0.25 \un{s}$ may still be feasible, but it displays significant overlapping between the two modes around $\sigma = 50 \un{nA}$.

Now we need to find the minimum sampling frequency that can distinguish the spraying modes. We'll fix $T_S = 0.5 \un{s}$ and compare the calculated 
statistical parameters for the following values of $f_s$: $f_s = 0.5 \un{kHz}; 1 \un{kHz}; 2 \un{kHz}; 5 \un{kHz}$. \autoref{fig:third-test-subsample}
shows the result obtained for this test.

\begin{figure}[h!]
    \centering
    \includegraphics[width=1\textwidth,trim=1 1 1 1,clip]{figures/third-test-subsample.png}
    \caption{Calculated standard deviation for $T_s = 0.5 \un{s}$, with $f_s = 0.5 \un{kHz}; 1 \un{kHz}; 2 \un{kHz}; 5 \un{kHz}$}
    \label{fig:third-test-subsample}
\end{figure}

As we can see on \autoref{fig:third-test-subsample}, $f_s = 2 \un{kHz}$ appears to the be the smallest sampling frequency that can still distinguish the spraying  
modes. 

The conclusion we can derive from these tests is that $T_S = 0.5 \un{s}$ and $f_s = 2 \un{kHz}$ are the mininum sample size and sampling frequency that
can distinguish the spraying modes via statistical parameters. Note that this results in a sample size of $N_s = T_S \cdot f_s = 1.000$, which uses 
significantly less memory than the $N_s = 20.000$ used on previous sections. We'll move forward with these values of $N_s$ and $f_s$, seeking to test the classification
with parameters that are consistent with Gilbert's requirements.

\subsection{Optimizing the Classification}

\subsubsection{Trying different statistical parameters}

So far, we've only the standard deviation of the signal to differentiate the spraying modes. However, we can also use other statistical parameters 
to distinguish the waveforms. The first one that we can try is the Relative Standard Deviation (RSD), defined on \autoref{eq:rsd}


\begin{equation} \label{eq:rsd}
    RSD = \left|\frac{\sigma}{\overline{I}}\right|
\end{equation}

where

\begin{itemize}
    \item $\sigma$: standard deviation
    \item $\overline{I}$: arithmetic mean
\end{itemize}

\autoref{tab:rsd} shows how the RSD can be a useful metric in the classification. When the spraying mode is intermittent, we 
expect the signal to display a large $\sigma$ and a small $\overline{I}$, since the potential is lower and therefore the mean value 
of the current in the system also is lower. This results in an overall value large value of the ratio. 

\begin{table}[h!]
    \begin{center}
      \caption{Expected behaviour of RSD for different spraying modes.}
      \label{tab:rsd}
      \begin{tabular}{c|c|c}
        \textbf{} & \textbf{Intermittent} & \textbf{Cone-jet}\\
        \hline
        Numerator ($\sigma$) & HIGH & LOW\\
        Denominator ($\overline{I}$) & LOW& HIGH\\
        Overall Ratio ($\sigma / \overline{I}$) & LARGE & SMALL\\
      \end{tabular}
    \end{center}
  \end{table}

On the other hand, when we have a cone-jet, we expect the signal to display a small $\sigma$ - as the signal is much more stable -
and a larger $\overline{I}$, given the larger potential. This results in an overall value small value of the ratio.

Therefore, both components of the fraction contribute in opposite directions to change the overall value of the ratio between the spraying modes,
making this metric a potentially good classification parameter.

\autoref{fig:rsd-subsample} shows the the same result of \autoref{fig:third-test-subsample} using the RSD instead of the standard deviation.

\begin{figure}[h!]
    \centering
    \includegraphics[width=\textwidth,trim=1 1 1 1,clip]{figures/rsd-subsample.png}
    \caption{Calculated RSD for $T_s = 0.5 \un{s}$, with $f_s = 0.5 \un{kHz}; 1 \un{kHz}; 2 \un{kHz}; 5 \un{kHz}$}
    \label{fig:rsd-subsample}
\end{figure}

As we can see on \autoref{fig:rsd-subsample}, we can achieve a good distinction between the spraying modes with the RSD. However, the 
order of magnitude of the value is very small, which can be inconvenient. A simple to resolve this is to take the inverse of the RSD, that 
we can define as the Signal-to-Noise Ratio (SNR), shown on \autoref{eq:snr}

\begin{equation} \label{eq:snr}
    SNR = \frac{1}{RSD} = \left|\frac{\overline{I}}{\sigma}\right|
\end{equation}

In \autoref{eq:snr}, we call the standard deviation as the "noise", and the mean as the "signal". \autoref{fig:snr-subsample} shows the
same result of \autoref{fig:third-test-subsample} using the SNR.

\begin{figure}[h!]
    \centering
    \includegraphics[width=\textwidth,trim=1 1 1 1,clip]{figures/snr-subsample.png}
    \caption{Calculated SNR for $T_s = 0.5 \un{s}$, with $f_s = 0.5 \un{kHz}; 1 \un{kHz}; 2 \un{kHz}; 5 \un{kHz}$}
    \label{fig:snr-subsample}
\end{figure}

As seen on \autoref{fig:snr-subsample}, the SNR is also a good metric to distinguish the intermittent and cone-jet modes.
Unlike the RSD, the SNR spreads from the range 0 - 50, making it more convenient than the RSD and showing less outliers. 

\subsubsection{Classification via small Neural Networks}

\hl{ASK BEN}

\subsection{Proof-of-Concept Real-time EHDA Classification}\label{sec:classification-algorithm}

Now that we know how to acquire the signal, we can develop a Proof-of-Concept (POC)
algorithm that classifies the spraying mode in real-time by looking at the current signal.
Algorithm \ref{alg:classification} shows a pseudocode for the algorithm developed.

\begin{algorithm}[!htbp]
    \caption{Real-time EHDA Classification.}\label{alg:classification}
    \begin{algorithmic}[1]
    \LineComment{This procedure obtains the raw data fom the oscilloscope and filters it, returning it to the caller}
    \Procedure{GetOscilloscopeMeasurement}{\null}
        \State raw\_data $\gets$ Array[$n$] \Comment{$n$ is the sample size, defined as $n = T_S \cdot f_s$.}
        \State raw\_data $\gets$ \Call{GetOscilloscopeData}{\null} \Comment{$\bigO (n)$}
        \Statex
        \LineComment{FFT - Fast Fourier Transform. $\bigO (n\log{}n)$}
        \State fft\_result $\gets$ \Call{GetFFT}{raw\_data} 

        \LineComment{Remove unwanted frequencies from the frequency spectrum.}
        \State low\_pass\_cutoff $\gets$ $100$
        \State bandstop\_center $\gets$ $50$
        \State bandstop\_width $\gets$ $2$
        \State bandstop\_start $\gets$ bandstop\_center - bandstop\_width
        \State bandstop\_end $\gets$ bandstop\_center + bandstop\_width

        \ForAll {f \textbf{in} fft\_result}
            \If{$f \ge$ low\_pass\_cutoff}
                \State $f \gets 0$ 
            \EndIf
            \If{$f >$ bandstop\_start \textbf{or} $f <$ bandstop\_end}
                \State $f \gets 0$ 
            \EndIf
        \EndFor
        \Statex

        \LineComment{Rebuild signal using IFFT - Inverse FFT. $\bigO (n\log{}n)$}
        \State recovered\_signal $\gets$ \Call{GetIFFT}{fft\_result} 
        \State \textbf{return} recovered\_signal

    \EndProcedure 
    \Statex

    \Procedure{ClassifyEHDA}{filtered\_data}
        \State sd\_threshold $\gets 70 \cdot 10^{-9}$ \Comment{Threshold for the standard deviation}
        \State calculated\_sd $\gets$ \Call{CalcStdDeviation}{filtered\_data} \Comment{$\bigO (n)$}
        \If{calculated\_sd $>$ sd\_threshold}
            \State \textbf{return} INTERMITTENT
        \Else
            \State \textbf{return} CONE\_JET
        \EndIf
    \EndProcedure 
    \Statex

    \While{$1$}
        \State filtered\_data $\gets$ \Call{GetOscilloscopeMeasurement}{\null} \Comment{$\bigO (n\log{}n)$}
        \State classification $\gets$ \Call{ClassifyEHDA}{filtered\_data} \Comment{$\bigO (n)$}

        \If{classification $==$ CONE\_JET}
            \State print(CONE\_JET)
        \Else
            \State print(INTERMITTENT)
        \EndIf
    \EndWhile
    \end{algorithmic}
\end{algorithm}

Algorithm \ref{alg:classification} consists of two helper functions - \codeword{GetOscilloscopeMeasurement}
and \codeword{ClassifyEHDA} - and a main loop. 
The former is responsible for acquiring and filtering the signal of $i_{GND}$ read,
while the latter classifies the filtered data based on the standard deviation.
For simplicity, we used only the standard deviation to classify the spray mode, 
but the SNR can also be used as discussed previously. 
The main loop continuouly acquires the data and classifies it.

Do notice that the function that acquires the raw data is highly dependant on the language and hardware 
used, therefore it was not detailed in the pseudocode. We used a TiePie oscilloscope, which
provides an API in Python.

For the complexity of the algorithm, the slowest operations are the Fast Fourier
Transform algorithms, with complexity $\bigO (n\log{}n)$. Therefore, the complexity 
of the entire algorithm is $\bigO (n\log{}n)$, where $n = T_S \cdot f_s$ i.e. $n$ is the 
sample size of the measurement.

To optimize the execution of the FFT and IFFT algorithms, we can use $n$ (the sample size) as a power of two.
In our case, if we want $T_S = 500 \un{ms}$ with $f_s = 2 \un{kHz}$, then we can pick $n = 1024$, which is almost equal to a 
500 ms time window and remains a power of two to optimize the Fourier algorithms.

\autoref{fig:real-time-ehda-classification}
shows the result of Algorithm \ref{alg:classification} implemented in Python.

\begin{figure}[h!]
    \centering
    \includegraphics[width=\textwidth,trim=1 1 1 1,clip]{figures/real-time-ehda-classification.png}
    \caption{Result of Algorithm \ref{alg:classification} implemented in Python.}
    \label{fig:real-time-ehda-classification}
\end{figure}

As we can see on \autoref{fig:real-time-ehda-classification}, when all the nozzles operate in
the cone-jet mode, the acquired signal is stable and displays a small standard deviation.
On the other hand, when the nozzles are intermittent, the acquired signal displays a 
large standard deviation, which is used for the classification.

\autoref{fig:real-time-ehda-classification-issues} shows some issues identified, that we need to keep 
in mind during the classification.

\begin{figure}[h!]
    \centering
    \includegraphics[width=\textwidth,trim=1 1 1 1,clip]{figures/real-time-ehda-classification-issues.png}
    \caption{Issues identified on Algorithm \ref{alg:classification} implemented in Python.}
    \label{fig:real-time-ehda-classification-issues}
\end{figure}

Firstly, we see that the signal can display "steps", going from one average value of current to another 
as we see on \autoref{fig:real-time-ehda-classification-issues}. This leads to a large calculated standard 
deviation, causing the classification to be intermittent despite all the nozzles operating in a
cone-jet mode. This can be solved by discarding measurements where the mean value of the current 
is different on the start and end of the window.

Secondly, it was observed that the signal occasionally did not bahave as expected. There were several instances 
where the acquired signal would display a much larger standard deviation than expected, and then immediately 
return to the expected pattern. To reduce the impact of these unexpected instances, we can use a classification 
based on subsequent windows: instead of looking at only the signal acquired on the current time window, 
we look at the previous 5 classifications as well. If the majority of the previous classifications is also cone-jet,
then we continue to classify the current time window as cone-jet.

This stategy has the main drawback of being significantly slower than looking at just the current time window. When the spray 
mode indeed changes from intermittent to cone-jet, we need to wait for the previous time windows to update as well.
We attempted to reduce the time window to $T_S = 250 \un{ms}$ to make this option viable, but with such small time
windows the calculated standard deviation did not allow for a reliable classification, 
like we had seen in \autoref{fig:first-test-subsample}. $T_S = 500 \un{ms}$ was the minimum $T_S$ that the algorithm still worked
reliably.

During this time, we faced significant challenges to reproduce the results on 
\autoref{fig:real-time-ehda-classification} with the sprayhead. We believe this could have 
been caused by degradation of the sprayhead and its needles. From now on we used
a new sprayhead, with which we spray upside down, and used $f_s = 5 \un{kHz}$ with a time
window of $T_S = 1 \un{s}$, since we saw we could reproduce the results more easily 
with this configuration.

\subsection{Proof-of-Concept Real-Time EHDA Control}

We can make a small change on Algorithm \ref{alg:classification} to implement small corrections on $V_+$ if the signal 
is classified as intermittent, as shown on Algorithm \ref{alg:control} below. Note that we have not shown all procedures as
we did on Algorithm \ref{alg:classification}, since they've remained unchanged.

\begin{algorithm}[h!]
    \caption{Real-time EHDA Control.}\label{alg:control}
    \begin{algorithmic}[1]
        \State voltage\_step $\gets 400$  
        \While{$1$}
            \State filtered\_data $\gets$ \Call{GetOscilloscopeMeasurement}{\null}
            \State classification $\gets$ \Call{ClassifyEHDA}{filtered\_data}
            \State applied\_voltage $\gets$ \Call{GetAppliedVoltage}{\null}

            \If{classification $==$ INTERMITTENT}
                \State \Call{SetAppliedVoltage}{applied\_voltage + voltage\_step}
            \EndIf
        \EndWhile
    \end{algorithmic}
\end{algorithm}

The voltage step of $400 \un{V}$ was determined experimentally. Note that we only did simple tests for this POC, more 
tests are necessary to find the ideal value the voltage step. \autoref{fig:real-time-ehda-control} shows the result 
obtained of Algorithm \ref{alg:control} implemented in Python.

\begin{figure}[h!]
    \centering
    \includegraphics[width=\textwidth,trim=1 1 1 1,clip]{figures/real-time-ehda-control.png}
    \caption{Result of Algorithm \ref{alg:control} implemented in Python.}
    \label{fig:real-time-ehda-control}
\end{figure}

On \autoref{fig:real-time-ehda-control}, for the first 10 seconds we keep $V_+$ fixed at $V_+ = 5500 \un{V}$ to stabilize the system.
On $t = 10 \un{s}$ we drop the voltage by 1 kV to cause an intermittent spraying mode and then we activate the control algorithm. 
As we see on the first image on the top, once the nozzles are intermittent the calculated standard deviation is above the threshold of
70 nA. This causes the control algorithm to increase $V_+$ by steps of 400 V until the standard deviation is below the threshold - which
corresponds to a cone-jet spraying mode seen on the bottom image.
Once this is achieved, the controller stops to increase the voltage, keeping it fixed.  

As we see on \autoref{fig:real-time-ehda-control}, the controller took over 5 seconds to go from the intermittent to the cone-jet, which 
may not be sufficient for Gilbert's requirements. However, this is just a Proof-of-Concept to illustrate the general idea of the algorithm,
and ideally we would place this algorithm on the embedded computer used on Gilbert's product to verify how it would behave in the final
application.

\section{PID Controller Investigations}\label{sec:pid}

Based on the mappings done for the Work Package 1, given certain boundaries of $V_+$, we can know if the spraying mode of the nozzles
is a cone-jet if the mean value of $i_N$ is in a certain range. Therefore, a first and simple attempt that we make to control the EHDA
spray mode of the nozzles is ensure that $i_N$ is always within this this stable zone.

To achieve this, we can use a PID controller to ensure $i_N$ is always in the center of the stable zone. The designed control loop 
is shown on \autoref{fig:block-diagram-pid}. The control system is provided a setpoint, which is value of $i_N$ that it will attempt
to keep fixed. In this sense, the controller will reject disturbances in order to keep $i_N$ always equal to the setpoint.
Therefore, it would ensure that the sprayhead is operating the in stable region mapped on WP1. 

\begin{figure}[h!]
    \centering
    \includegraphics[width=.8\textwidth,trim=1 1 1 1,clip]{figures/block-diagram-pid.png}
    \caption{Block diagram of control loop designed.}
    \label{fig:block-diagram-pid}
\end{figure}

In \autoref{fig:block-diagram-pid}, the input of the system is the mean value of $i_{N_{act}}$, the actual value $i_N$ currently
in the system, sampled at $f_s = 5 \un{kHz}$ over a time window
$T_S = 0.5 \un{s}$. The input $i_{N_{act}}$ is subtracted from the previous value of $i_{N_{prev}}$, producing the error signal $e$. 
The error in the input to the controller block $C$, which receives the error as input and outputs a control voltage $V_+$ that the 
power supply must apply to restore $i_N$ to the setpoint. This voltage is applied to the sprayhead - block $SH$ - , producing a new mean value 
of $i_{N_{act}}$ which is provided again as input to the system, repeating the process.

How fast the controller behaves depends greatly of value chosen for its gains: the proportional gain $k_p$, the derivative gain $k_d$
and the integral gain $k_i$. For initial tests, we'll attempt to find the best value for these parameters via simple trial and error,
in order to get a general idea if this controller would work for this application. 

The setup used in this test is shown on \autoref{fig:setup-pid}. We read the current from the power supplies connected 
to the computer running the algorithm, and therefore no oscilloscopes are necessary. We use the HS camera with microscopic lens to verify 
visually the EHDA spraying mode of the nozzles.

\begin{figure}[h!]
    \centering
    \includegraphics[width=\textwidth,trim=1 1 1 1,clip]{figures/setup-pid.png}
    \caption{Setup used for PID controller tests.}
    \label{fig:setup-pid}
\end{figure}

The result obatined is shown on \autoref{fig:result-pid}.

\begin{figure}[h!]
    \centering
    \includegraphics[width=\textwidth,trim=1 1 1 1,clip]{figures/result-pid.png}
    \caption{Results of PID controller tests.}
    \label{fig:result-pid}
\end{figure}

As we can see on \autoref{fig:result-pid}, this initial version of the PID controller is able to autonosmoly adjust $V_+$ in order
for $i_N$ to reach the setpoint, which in this case was the center of stable region.
However, we can see it took about 8 seconds for the current and voltage levels to stabilize.
The response time can be improved - i.e. make the controller faster - by changing the gains of the controller, 
but doing this by trial and
error is not feasible. The correct method is to experimentally model the transfer function of the sprayhead in terms 
of its inputs and outputs, and use this function to obtain the ideal gains based on the time-response requirements. 

This method is very time-consuming and complex, and largely falls outside our area of expertise. Therefore, we focused
more on the current-based classification and control method discussed on Section \ref{sec:current-based}. Nevertheless, 
given more time to properly model the transfer function of the sprayhead, the PID controller could be a good candidate 
for ensuring $i_N$ is always at the stable region.

\section{Other Results}\label{sec:utils}

\subsection{Flow Rate disturbances}

During the PID Controller investigations, we had to analyse possible disturbances that the controller would have to reject.
One possible disturbance would be disturbances in $\phi$, which Gilbert told us could be in the range of $\pm 10\%$ in the
final device. Therefore, using the same setup of \autoref{fig:setup-pid}, we would begin with $\phi = 30 \un{mL/h}$, and then 
change the flow rate of the syringe pump mid-operation in two cases:

\begin{enumerate}
    \item $+ 100\% \,\, \phi$: this is an extreme case do verify the general behaviour of the nozzles with a large change in $\phi$
    \item $\pm 10\%  \,\,\phi$: the case given by Gilbert
\end{enumerate}

We used the HS Camera to record and visualize the spraying mode of the nozzles before and after the introduction of the disturbance.
The results obtained are shown on Figures \ref{fig:100-disturbance-start} and \ref{fig:100-disturbance-end}.

\begin{figure}[h!]
    \centering
    \includegraphics[width=\textwidth,trim=1 1 1 1,clip]{figures/100-disturbance-start.png}
    \caption{Behaviour of $i_N$ and cone-jets during introduction of $\phi$ disturbance.}
    \label{fig:100-disturbance-start}
\end{figure}

On \autoref{fig:100-disturbance-start} (a), the flow rate is $\phi = 30 \un{mL/h}$ and $i_N = 2 \un{uA}$. On $t = 5 \un{s}$, 
we change the flow rate to $\phi = 60 \un{mL/h}$, corresponding to an $+ 100\%$ change in $\phi$. As we can see on 
\autoref{fig:100-disturbance-start} (b), the current increases to $i_N = 3 \un{uA}$, the cone-jet become visibliy more elongated 
but they are still stable, as well as the mist. The current remains stable in this new value until we return $\phi$ back to its
original value on $t = 20 \un{s}$, as shown on \autoref{fig:100-disturbance-end} (a).

\begin{figure}[h!]
    \centering
    \includegraphics[width=\textwidth,trim=1 1 1 1,clip]{figures/100-disturbance-end.png}
    \caption{Behaviour of $i_N$ and cone-jets during removal of $\phi$ disturbance.}
    \label{fig:100-disturbance-end}
\end{figure}

On \autoref{fig:100-disturbance-end} (b), both the cone-jets and $i_N$ return to as they were on \autoref{fig:100-disturbance-start} (a).

These results show that the sprayhead is very robust when it comes to flow rate disturbances. We repeated the tests for $\pm 10 \%$ changes
in $\phi$ and both $i_N$ and the cone-jets barely change at all.

\subsection{Relations between $i_N$, $i_{GND}$ and $i_C$}

During the experiments of \autoref{fig:three-currents-spray-mode-50ms}, where the three currents of the system
were measured, we found a relation between the values of
$i_N$, $i_{GND}$ and $i_C$. It appeared that $i_C$ was the sum of $i_{GND}$ and $i_N$, in the direction defined at
\autoref{fig:setup-conventions}. In this case, we'd find that

\begin{equation} \label{eq:sum-ic}
    i_C = i_N + i_{GND}
\end{equation}

In other to experimentally verify \autoref{eq:sum-ic}, we defined two parameters:

\begin{equation} \label{eq:diff}
    DIFF = i_- - \left(i_{GND} + i_N\right)
\end{equation}

\begin{equation} \label{eq:diff-percent}
    DIFF = \frac{i_- - \left(i_{GND} + i_N\right)}{i_-} \cdot 100\%
\end{equation}

Both parameters on Equations \ref{eq:diff} and \ref{eq:diff-percent} measure how close the difference of both
sides of Equation \ref{eq:sum-ic} is to zero,
which should be case if Equation \ref{eq:sum-ic} is true. Calculating the value of theses parameters on the data of \autoref{fig:three-currents-spray-mode-50ms}
results in \autoref{tab:table-diff}.

\begin{table}[h!]
    \begin{center}
        \caption{Calculated $DIFF$ parameters for the collected data of three currents}
        \includegraphics[width=.6\textwidth,trim=1 1 1 1,clip]{figures/table-diff.png}
        \label{tab:table-diff}
    \end{center}
  \end{table}

As we can see on \autoref{tab:table-diff}, $i_N + i_{GND}$ is always within $5\%$ of the absolute value of $i_C$, regardless of potential,
flow rate or EHDA spray mode. We had discussions as to whether 
or not this could be a useful metric for the control or the charge neutralization measurements, but we did not explore it further. We leave it here 
if it is something that Gilbert wishes to explore in the future. 

\subsection{Inrush Current on $i_N$}

During the work, Gilbert requested us to verify the spike of current on $i_N$ when switching on $V_+$. To do this, we used the same setup 
of \autoref{fig:setup-three-currents-spray-mode}, only measuring the current $i_N$ with the oscilloscope. $V_-$ was fixed on $V_- = - 4.5 \un{kV}$, 
with $\phi = 30 \un{mL/h}$. The oscilloscope was configured for $f_s = 50 \un{kHz}$ to capture the fast response of the inrush current.
$V_+$ was switched on from $V_+ = 0$ to $V_+ = 5 kV$, while $V_-$ was already swicthed on. \autoref{fig:inrush-current} shows the result 
obtained in this test.

\begin{figure}[h!]
    \centering
    \includegraphics[width=\textwidth,trim=1 1 1 1,clip]{figures/inrush-current.png}
    \caption{Inrush current measured on $i_N$ (a) over a 5 second time window and (b) with a zoom on the initial spike.}
    \label{fig:inrush-current}
\end{figure}

\autoref{fig:inrush-current} (a) shows that we have a short spike in the current, better viewed on \autoref{fig:inrush-current} (b).
The initial spike lasts for about 6 ms and has a peak value of $i_N = 4 \un{uA}$. About 1 second after switching $V_+$ on, the
current achieves its peak overshoot value of $i_N = 4.5 \un{uA}$. 2.5 seconds after switching $V_+$ on the current stabilizes 
on its final value of $i_N = 3 \un{uA}$.

Both peaks of current \autoref{fig:inrush-current} (a) are about 50\% higher than the final value of $i_N$, and well below the 
safety threshold of 10uA given by Gilbert. We did not see a large difference on $i_N$ when switching both $V_+$ and $V_-$ on at the 
same time. 

Notice that the FUG HV power supply accused a very large inrush current on its display, as shown on \autoref{fig:inrush-current-fug}.
However, this value contradicts the value given by the oscilloscope, which is a better measurement instrument. In addition, if
indeed we had $i_N = 59 \un{uA}$ as the FUG displays, it would have resulted in a voltage drop of $\Delta V = 118 \un{V}$ accross the 
oscilloscope in series with $V_+$, which would have fried the channel according to instrument's datasheet. This was not the case, 
and therefore the physical evidence suggests that the inrush current in indeed as low as the oscilloscope shows. 

\begin{figure}[h!]
    \centering
    \includegraphics[width=\textwidth,trim=1 1 1 1,clip]{figures/inrush-current-fug.png}
    \caption{Inrush current displayed by the FUG HV power supply.}
    \label{fig:inrush-current-fug}
\end{figure}

\subsection{Switching $i_N$ ON and OFF in a 2-second interval}

Gilbert also requested us to verify the behaviour of $i_N$ over several switches on $V_+$ in an interval of 2 seconds. We used the same 
setup described for \autoref{fig:inrush-current}, obtaining the result shown on \autoref{fig:on-off-2-seconds}.

\begin{figure}[h!]
    \centering
    \includegraphics[width=\textwidth,trim=1 1 1 1,clip]{figures/on-off-2-seconds.png}
    \caption{Behaviour of $i_N$ when switching $V_+$ on and off in a 2-second interval.}
    \label{fig:on-off-2-seconds}
\end{figure}

As seen on \autoref{fig:on-off-2-seconds}, the inrush current peak is higher for the first time $i_N$ is swicthed on.
After the first time, it is about 50\% smaller. The overshoot peak, however, increses as the system is switched on and off several times.
We believe this is caused by the wetness that builds up on the nozzles. 


% ==============================================
\newpage    \pagestyle{plain}
\addcontentsline{toc}{chapter}{References} % 'References' will appear in the index.
\bibliography{references.bib}
% ==============================================
\end{document}